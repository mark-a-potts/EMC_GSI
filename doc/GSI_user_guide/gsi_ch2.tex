\chapter{Software Installation}
\setlength{\parskip}{12pt}

%-------------------------------------------------------------------------------
\section{Introduction}
%-------------------------------------------------------------------------------

The DTC community GSI is a community distribution of NOAA\textquotesingle s operational GSI. 
The community GSI expands the portability of the operational code by adding a flexible build 
system and providing example run scripts that allow GSI to be compiled and run on many common 
platforms. The current version of GSI is 3.5. It builds and runs on most standard
Linux platforms using either Intel, PGI, and Gnu compilers. Legacy build rules are provided for 
two platforms, the IBM AIX computers using the xlf compiler, and Intel based Macintosh 
computers using the PGI compiler. In both cases, the default build system must be modified to 
build on these platforms. See the community web page user support FAQ to get started.

This chapter describes how to build and install the DTC community GSI software on your local 
Linux computing resources. These instructions apply only to the DTC community GSI. While 
the community GSI source code is identical to the tag of the NCEP\textquotesingle s GSI 
trunk code used for release, the community build system is different, allowing it to be more 
general to support a wide variety of computing platforms.

The GSI building process consists of four general steps:
\begin{itemize}
\item Obtain the source code for GSI and WRF.
\item Build the WRF model (see the WRF users guide).
\item Set the appropriate environment variables for the GSI build.
\item Configure and compile the GSI source code.
\end{itemize}

This chapter is organized as follows: Section \ref{ch2_obtainingcode} describes how to obtain 
the source code. Section \ref{ch2_directorystructure} covers the directory structure and 
supplemental NCEP libraries included with the distribution. Section \ref{ch2_compiling} starts 
with an outline of the build example and then goes into a more detailed discussion of setting 
up the build environment and the configure and compile steps. 
Section \ref{ch2_buildexample} illustrates the build process for the three of the compilers (Intel, 
PGI and Gnu) on the NCAR supercomputer Yellowstone. 
Section \ref{ch2_externallibs} covers the system requirements and settings (tools, libraries, and environment 
variable settings), and currently supported platforms in detail. 
Section \ref{ch2_gettinghelp} discusses what to do if you have problems with the build and where to get help.

For beginning users, sections \ref{ch2_obtainingcode} and \ref{ch2_compiling} provide the necessary 
steps to obtain the code and build GSI on most systems. The remainder of the chapters provide 
background material for completeness. 
Advanced topics, such as customizing the build, porting to new platforms, and debugging can be 
found in the GSI Advanced User\textquotesingle s Guide.

%-------------------------------------------------------------------------------
\section{Obtaining and Setting Up the Source Code}  \label{ch2_obtainingcode}
%-------------------------------------------------------------------------------

The community GSI resources, including source code, build system, utilities, practice data, and 
documentation, are available from the DTC community GSI users website, located at

\url{http://www.dtcenter.org/com-GSI/users/index.php}

The source code is available by first selecting the \underline{Download} tab on the vertical 
menu located on the left column of the page, and then selecting the 
\underline{GSI/EnKF System} 
submenu. New users must first register before downloading the source code. Returning users 
only need to enter their registration email address to log in. After accessing the download page, 
select the link to the \verb|comGSIv3.5_EnKFv1.1.tar| tarball to download the most recent version 
of the source code (July 2016). Selecting the newest release of the community GSI is critical for 
having the most recent capabilities, versions of supplemental libraries, and bug fixes. Full support 
is only offered for the two most recent code releases.

To analyze satellite radiance observations, GSI requires use of CRTM coefficients. It is important to
use \textbf{only} the version of CRTM coefficients provided by GSI website. Due to their large size, 
these are available as a separate tarfile. They can be downloaded by selecting the link to the 
tarball for the \verb|CRTM 2.2.3 Big Endian coefficients| from the web page. For all compilers 
use the big endian byte order coefficients found in the first CRTM link.

The download page also contains links to the fixed files necessary for 
running global GSI:

\begin{itemize}
\item Global configuration (fix files to run Global GSI)
\end{itemize}

The community GSI version 3.5 comes in a tar file named \verb|comGSIv3.5_EnKFv1.1.tar|. The tar 
file may be unpacked by using the UNIX commands:

\begin{small}
\begin{verbatim}
gunzip comGSIv3.5_EnKFv1.1.tar.gz
tar -xvf comGSIv3.5_EnKFv1.1.tar
\end{verbatim}
\end{small}

This creates the top level GSI directory \verb|comGSIv3.5_EnKFv1.1/|.
After downloading the source code, and prior to building, the user should check the known issues 
link on the download page of DTC website to determine if any bug fixes or platform specific customizations are needed.

%-------------------------------------------------------------------------------
\section{Directory Structure, Source Code and Supplemental Libraries}  \label{ch2_directorystructure}
%-------------------------------------------------------------------------------

The GSI system includes the GSI source code, the build system, supplemental libraries, fixed files, and run scripts. 
The following table lists the system components found inside of the root GSI directory.


\begin{table}[htbp]
\centering
\begin{tabular}{| l | l |}
\hline
Directory Name  &   Content   \\
\hline
\hline
src/main/ & GSI source code and makefiles \\
\hline
src/libs/ & Source code for supplemental libraries \\
\hline
fix/     & Fixed input files required by a GSI analysis, such as  \\
         & background error covariances, observation error tables;  \\
         & excluding the CRTM coefficients \\
\hline
include/ & Include files created by the build system \\
\hline
lib/     & Location for built supplemental libraries \\
\hline
run/     & Directory for executable \verb|gsi.exe| and sample run scripts \\
\hline
arch/    & Build system support such as machine architecture specifics  \\
         & (see Advanced GSI User\textquotesingle s Guide) \\
\hline
util/    & Tools for GSI diagnostics \\
\hline
\end{tabular}
\label{ch2_tble1}
\end{table} 

For the convenience of the user, supplemental NCEP libraries for building GSI are 
included in the \verb|src/libs/| directory. These libraries are built when GSI is built. 
These supplemental libraries are listed in the table below.

\begin{table}[htbp]
\centering
\begin{tabular}{| l | l |}
\hline
\hline
Directory Name &  Content \\
\hline
bacio/  &  NCEP BACIO library  \\
\hline
bufr/   &  NCEP BUFR library  \\
\hline
crtm\_2.2.3/ &  JCSDA community radiative transfer model  \\
\hline
gsdcloud/   &  GSD Cloud analysis library  \\
\hline
misc/     &  Misc support libraries  \\
\hline
nemsio/  &  NEMS I/O library  \\
\hline
sfcio/  &  NCEP GFS surface file i/o module  \\
\hline
sigio/ &  NCEP GFS atmospheric file i/o module \\
\hline
sp/ &  NCEP spectral - grid transforms \\
\hline
w3emc\_v2.0.5   &  NCEP/EMC W3 library (date/time manipulation, GRIB) \\
\hline
w3nco\_v2.0.6  & NCEP/NCO W3 library (date/time manipulation, GRIB)  \\
\hline

\end{tabular}
\label{ch2_tble2}
\end{table} 

The one nonstandard \textit{library} not included with the source code are the WRF IO API\textquotesingle s. These are obtained by linking to a build of the WRF code. Please note that the release version of WRF/EnKF has only been tested using the previous two release versions of WRF. Older versions of WRF may provide unpredictable results.

The WRF code, and full WRF documentation, can be obtained from the WRF Users\textquotesingle \ Page,

\url{http://www.mmm.ucar.edu/wrf/users/}

following a registration process similar to that for downloading GSI.

%-------------------------------------------------------------------------------
\section{Compiling GSI} \label{ch2_compiling}
%-------------------------------------------------------------------------------

This section starts with a quick outline of how to build GSI (\ref{ch2_compiling_overview}), followed by a more detailed discussion of the build process (\ref{ch2_compiling_env} \& \ref{ch2_compiling_configandcompile}). Typically GSI will build \textit{straight out of the box} on any system that successfully builds the WRF model. Should the user experience any difficulties with the default build, check the build environment against the requirements described at the end of section \ref{ch2_externallibs}.

To proceed with the GSI build, it is assumed that the WRF model has already been built on the current system. GSI uses the WRF I/O API libraries to read the background file. These I/O libraries are created as part of the WRF build, and are linked into GSI during the GSI build process. In order to successfully link the WRF I/O libraries with the GSI source, it is crucial that both WRF and GSI are built using the same Fortran compilers. This means that if WRF is built with the Intel Fortran compiler, then GSI must also be built with the Intel Fortran compiler. It is also recommended that both codes be built with the same annual version number of the compiler
(.e.g. 12, 13, 14, 15, etc.). 

%-------------------------------------------------------------------------------
\subsection{Build Overview} \label{ch2_compiling_overview}
%-------------------------------------------------------------------------------

This section provides a quick outline of the steps necessary to build the GSI code. The following steps describe that build process.

\begin{enumerate}
\item \textbf{Set the environment for the compiler:} If not already done so, set the necessary paths for using your selected compiler, such as loading the appropriate modules or modifying your path variable.
\item \textbf{Set the environment variables:} The first path on this list will always need to be set. The remaining two will depend on your choice of compiler and how your default environment is configured.
\begin{description}
\item[]a. \verb|WRF_DIR| the path to the compiled WRF directory (to always be set) 
\item[]b. \verb|NETCDF| the path to the \verb|NETCDF| libraries
\item[]c. \verb|LAPACK_PATH| the path to the \verb|LAPACK| math libraries
\end{description}
\item \textbf{Run the configure script}
\item \textbf{Run the compile script}
\end{enumerate}

%-------------------------------------------------------------------------------
\subsection{Environment Variables} \label{ch2_compiling_env}
%-------------------------------------------------------------------------------

Before configuring the GSI code to be built, at least one, and no more than three environment variables must be set.

\begin{description}
\item[WRF\_DIR] defines the path to the root of the WRF build directory. Setting this is mandatory. This variable tells the GSI build system where to find the WRF I/O libraries. The process for setting the environment variables varies according to the login shell in use. To set the path variable WRF\_DIR for csh/tcsh, type;
\begin{verbatim}
setenv WRF_DIR /path_to_WRF_root_directory/  for csh or tcsh
export WRF_DIR=/path_to_WRF_root_directory/  for ksh or bash
\end{verbatim}
\item[NETCDF] The second environment variable specifies the local path to NetCDF library. The path location for NETCDF environment variable may be checked by using the command 
\begin{verbatim}
            echo $NETCDF
\end{verbatim}
If the command returns with the response that the variable is undefined, such as
\begin{verbatim}
            NETCDF: Undefined variable.
\end{verbatim}
it is then necessary to manually set this variable. If your system uses modules or a similar mechanism to set the environment, do this first. If a valid path is returned by the echo command, no further action is required.\\

\item[LAPACK\_PATH] defines the path to the LAPACK library. Typically, this variable will only need to be set on systems without a vendor provided version of LAPACK.
IBM systems typically come installed with the LAPACK equivalent ESSL library that links automatically. Likewise, the PGI compiler often comes with a vendor provided version of LAPACK that links automatically with the compiler. Experience has shown that the following situations make up the majority of cases where the LAPACK variable needed to be set:
\begin{itemize}
\item Linux environments using Intel Fortran compiler.
\item Building with Gfortran.
\item On systems where the path variables are not properly set.
\item On stripped down versions of the IBM AIX OS that lack the  ESSL libraries
\end{itemize}
Of the four, the first of these is the most common. The Intel compiler usually comes with a vendor provided mathematics library known as the \textit{Mathematics Kernel Libraries} or MKL for short. While most installations of the Intel compiler typically come with the MKL libraries installed, the ifort compiler does not automatically load the library. It is therefore necessary to set the LAPACK\_PATH variable to the location of the MKL libraries when using the Intel compiler. You may need to ask your system administrator for the correct path to these libraries.
\end{description}

On super-computing systems with multiple compiler options, these variables may be set as part of the module settings for each compiler. On the NCAR supercomputer Yellowstone, the Intel build environment can be specified through setting the appropriate modules. When this is done, the MKL library path is available through a local environment variable, MKLROOT. The LAPACK environment may be set for csh/tcsh with the Unix commands
\begin{small}
\begin{verbatim}
            setenv LAPACK_PATH $MKLROOT
\end{verbatim}
\end{small}
and for bash/ksh by
\begin{small}
\begin{verbatim}
            export LAPACK_PATH=$MKLROOT
\end{verbatim}
\end{small}

Once the environment variables have been set, the next step in the build process is to first run the configure script and then the compile script.

%-------------------------------------------------------------------------------
\subsection{Configure and Compile} \label{ch2_compiling_configandcompile}
%-------------------------------------------------------------------------------

Once the environment variables have been set, building the GSI source code requires two additional steps:
\begin{enumerate}
\item Run the configure script and select a compiler option.
\item Run the compile script
\end{enumerate}
Change into the \verb|comGSIv3.5_EnKFv1.1/| directory and issue the configure command: 
\begin{verbatim}
           ./configure
\end{verbatim}
The \verb|./configure| command uses user input to create a platform specific configuration file called 
\verb|configure.gsi|. The script starts by echoing the NETCDF and WRF\_DIR paths set in the previous 
section. It then examines the current system and queries the user to select from multiple build options.

For 64-bit Linux the options will be the following:
\begin{scriptsize}
\begin{verbatim}
Will use NETCDF in dir: /glade/apps/opt/netcdf/4.3.0/intel/12.1.5
Will use WRF in dir: /glade/p/work/mhu/wrf/WRFV3.5
------------------------------------------------------------------------
Please select from among the following supported platforms.

   1.  Linux x86_64, PGI compilers (pgf90 & pgcc)  (dmpar,optimize)
   2.  Linux x86_64, PGI compilers (pgf90 & gcc)  (dmpar,optimize)
   3.  Linux x86_64, PGI compilers (pgf90 & gcc) Supercomp (w/o -f90=SFC)  (dmpar,optimize)
   4.  Linux x86_64, PGI compilers (pgf90 & pgcc) Supercomp (w/o -f90=SFC)  (dmpar,optimize)
   5.  Linux x86_64, GNU compilers (gfortran & gcc)  (dmpar,optimize)
   6.  Linux x86_64, Intel/gnu compiler (ifort & gcc)	 (dmpar,optimize)
   7.  Linux x86_64, Intel compiler (ifort & icc)  (dmpar,optimize)
   8.  Linux x86_64, Intel compiler (ifort & icc), IBM POE (EXPERIMENTAL)   (dmpar,optimize)
   9.  Linux x86_64, Intel compiler (ifort & icc), SGI MPT (EXPERIMENTAL)  (dmpar,optimize)
\end{verbatim}
\end{scriptsize}

Looking at the list, there are two things to note. First is that the GNU C-compiler (gcc) may be 
paired with any of the other Fortran compilers. This allows the build to use the GNU C-compiler 
in place of the Intel (icc) or PGI (pgcc) C-compiler.

The second thing to notice is that there are separate build targets for vendor supplied versions 
of MPI such as IBM POE and SGI MPT. This was added due to some computing hardware vendors 
creating non-standard mpif90 wrappers for their vendor supplied version of MPI. If uncertain about 
which to choose, start by selecting the default option corresponding to the fortran compiler you wish 
to use (either 1,2,5,6, or 7). If that option fails with an error referencing a bad argument for mpif90,
only then try the options listed for use with \verb|Supercomp|, \verb|IBM POE|, or \verb|SGI MPT|.

On selecting an option, the process reports a successful configuration with the banner:
\begin{scriptsize}
\begin{verbatim}
        ------------------------------------------------------------------------ 
        Configuration successful. To build the GSI, type: compile 
        ------------------------------------------------------------------------
\end{verbatim}
\end{scriptsize}
Failure to get this banner means that the configuration step failed to complete. The most typical reason for a failure is an error in one of the paths set to the environment variables.

After selecting a build option, run the compile script:
\begin{small}
\begin{verbatim}
      ./compile >& compile.log
\end{verbatim}
\end{small}
It is recommended to capture the build information to a log file by redirecting the output incase it is necessary to diagnose build issues. 

To conduct a complete clean, which removes ALL built files in ALL directories, as well as the configure.gsi, type:
\begin{small}
\begin{verbatim}
     ./clean -a
\end{verbatim}
\end{small}
A complete clean is necessary if the compilation failed or if the configuration file is changed.

Following a successful compile, the GSI executable \verb|gsi.exe| can be found in the \verb|run/| directory.
If the executable is not found, check the compilation log file. If the build failed, search for the first instance of the word "Error" (with a capital "E") to locate the section of the log with the failure.

%-------------------------------------------------------------------------------
\section{Example of Build} \label{ch2_buildexample}
%-------------------------------------------------------------------------------

To illustrate the build process, the following section describes the steps necessary to build GSI on the NCAR supercomputer Yellowstone using the Intel compiler, the PGI compiler, and the Gnu compiler. Other platforms will be similar.

%-------------------------------------------------------------------------------
\subsection{Intel Build}
%-------------------------------------------------------------------------------

Steps to build GSI on Yellowstone using the Intel compiler:
\begin{enumerate}
\item Select the Intel compiler environment by using the module commands:
\begin{verbatim}
          module load intel
          module load impi mkl ncarcompilers ncarbinlibs netcdf
\end{verbatim}
These module commands have specified the compiler, mpi, the version of the LAPACK library (MKL) and the netcdf library.
\item For this case two of the paths must be set. The path to the WRF directory must always be specified, and the Intel Mathematics Kernal Library (MKL) will be used in place of the LAPACK library. Note that on Yellowstone, the variable MKLROOT is set to the path to the MKL libraries by loading the mkl module. To set the paths in a C-shell environment use:
\begin{verbatim}
          setenv WRF_DIR /PATH TO WRF DIRECTORY/
          setenv LAPACK_PATH $MKLROOT
\end{verbatim}
\item To run the configure script, type \verb|./configure| inside the top of the GSI directory. If the first three steps were completed successfully, a table of compiler options should appear. Select the desired compiler combination, which in this case is either 6 or 7. The alternative options (8 and 9) are needed for certain platforms that have a vendor supplied custom version of MPI. Try the default build options for MPI first, and only if it fails should the second option be used.
\item To compile the code, enter in a C-shell: \verb|./compile >& compile.log|. If the build completes successfully, an executable named \verb|gsi.exe| will be created in the \verb|./run| directory.
\end{enumerate}

%-------------------------------------------------------------------------------
\subsection{PGI Build}
%-------------------------------------------------------------------------------

Steps to build GSI on Yellowstone using the PGI compiler:
\begin{enumerate}
\item The PGI compiler environment is selected using the module commands:
\begin{verbatim}
module load pgi
module load impi ncarcompilers ncarbinlibs netcdf
\end{verbatim}
These module commands have specified the compiler, mpi, and the netcdf library.
\item For this case only the path to the WRF directory must be set. The PGI compiler comes with its own version of LAPACK that it finds automatically. It is not necessary to set the LAPACK path. In a C-shell environment use:
\begin{verbatim}
          setenv WRF_DIR /PATH TO WRF DIRECTORY/
\end{verbatim}
\item Similar to the Intel example, pick compiler options listed in a table. In this case, the desired compiler combination is either 3 or 4.
\item To compile the code, enter in a C-shell: \verb|./compile >& compile.log|. If the build completes successfully, an executable named \verb|gsi.exe| will be created in the \verb|./run directory|.
\end{enumerate}

%-------------------------------------------------------------------------------
\subsection{GNU Build}
%-------------------------------------------------------------------------------

Steps to build GSI on Yellowstone using the GNU compiler:
\begin{enumerate}
\item The GNU compiler environment is selected using the module commands:
\begin{verbatim}
module load gnu/5.3.0
module load ncarcompilers ncarbinlibs netcdf lapack/3.2.1 
\end{verbatim}
These module commands have specified the compiler, mpi, and the netcdf library.
\item For this case two of the paths must be set. The path to the WRF directory must always be specified, and we will use the LAPACK library installed by the module.  In a C-shell environment use:
\begin{verbatim}
          setenv WRF_DIR /PATH TO WRF DIRECTORY/
          setenv LAPACK_PATH $LIB_NCAR
\end{verbatim}
\item Similar to the Intel example, pick compiler options listed in a table. In this case, the desired compiler combination is 5.
\item To compile the code, enter in a C-shell: \verb|./compile >& compile.log|. If the build completes successfully, an executable named \verb|gsi.exe| will be created in the \verb|./run directory|.
\end{enumerate}

%-------------------------------------------------------------------------------
\section{System Requirements and External Libraries}  \label{ch2_externallibs}
%-------------------------------------------------------------------------------

The source code for GSI is written in FORTRAN, FORTRAN 90, and C. In addition, the parallel executables require some flavor of MPI and OpenMP for the distributed memory parallelism. Lastly the I/O relies on the NetCDF I/O libraries. Beyond standard shell scripts, the build system relies on the Perl scripting language and makefiles.

The basic requirements for building and running the GSI system are the following:
\begin{itemize}
\item FORTRAN 95+ compiler 
\item C compiler
\item MPI v1.2+
\item OpenMP
\item Perl
\item NetCDF V3.6.3 or V4.2+
\item LAPACK and BLAS mathematics libraries, or equivalent 
\item WRF V3.5+
\end{itemize}

Because all but the last of these tools and libraries are typically the purview of system administrators to install and maintain, they are lumped together here as part of the basic system requirements.

%-------------------------------------------------------------------------------
\subsection{Compilers Tested for Release}
%-------------------------------------------------------------------------------

Version 3.5 of the DTC community GSI system has been successfully tested on a variety of Linux platforms with many versions of the Intel and PGI fortran compilers.

Legacy build rules are also available for IBM AIX and Mac Darwin platforms. Because the DTC does not have the ability to test on these platforms, they are no longer supported. Also, Linux GNU gfortran option is added in this version.

The following Linux compiler combinations have been fully tested:
\begin{table}[htbp]
\centering
\begin{tabular}{| l | l | l |}
\hline
  & Fortran compiler version   &   C compiler version    \\
\hline
\hline
Intel only  &  ifort    16.0.1, 15.0.1, 13.0.1, 12.1.5, 12.1.4       & icc \\
\hline
Intel \& gcc  &  ifort    16.0.1, 15.0.1, 13.0.1, 12.1.5, 12.1.4       &  gcc 4.8.2, 4.4.7  \\
\hline
PGI only  &   pgf90   16.1, 15.10, 15.7, 15.1, 14.10, 14.9, 14.7, 13.9, 13.3    &  pgcc  \\
\hline
PGI \& gcc  &   pgf90   16.1, 15.10, 15.7, 15.1, 14.10, 14.9, 14.7, 13.9, 13.3    &    gcc 4.8.2 \\
\hline
GNU only &  gfortran  6.3.0, 5.3.0   & gcc 6.3.0, 5.3.0 \\
\hline
\end{tabular}
\label{ch2_tble3}
\end{table} 

Unforeseen build issues may occur when using older compiler and library versions. As always, the best results come from using the most recent version of compilers.

%-------------------------------------------------------------------------------
\section{Getting Help and Reporting Problems}  \label{ch2_gettinghelp}
%-------------------------------------------------------------------------------

Should the user experience any difficulty building GSI on their system, please first confirm that 
all the required software is properly installed (section \ref{ch2_compiling}). Next check that the 
external libraries exist and that their specified paths in the configure file are correct. Lastly, 
check the resource file \textit{configure.gsi} for errors in any of the paths or settings. Should 
all these check out, feel free to contact the community GSI Help Desk for assistance at 
\begin{center}
{gsi-help@ucar.edu}
\end{center}

At a minimum, when reporting code building problems to the helpdesk, please include with your 
email a copy of the build log, and the \textit{configure.gsi} file.

