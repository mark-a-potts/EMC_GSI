\begin{titlepage}
%\BgThispage
%\newgeometry{left=1cm,right=4cm}
\vspace*{0.5cm}
\noindent

\begin{flushleft}
\textcolor{darkgray}{\LARGE Forward}
\vspace*{1cm}\par

This document is the 2016 Gridpoint Statistical Interpolation (GSI) User\textquotesingle s Guide geared particularly for beginners. It describes the fundamentals of using GSI version (v) 3.5 released in July 2016. Advanced features of GSI as well as details of assimilation of specific data types can be found in the Advance GSI User\textquotesingle s Guide, released together with this document and the v3.5 code release.

This User\textquotesingle s Guide includes six chapters and three appendices:
\begin{description}
\item[Chapter 1] provides a background introduction of GSI.
\item[Chapter 2] contains basic information about how to install and compile GSI - including system requirements; required software (and how to obtain it); how to download GSI; and information about compilers, libraries, and how to build the code.
\item[Chapter 3] focuses on the input files needed to run GSI and how to configure and run GSI through a sample run script. Also provides example of a successful GSI run and explanations of often used namelist variables.
\item[Chapter 4] includes information about diagnostics and tuning of the GSI system through GSI standard output, statistic fit files, and some diagnostic tools.
\item[Chapter 5] illustrates the GSI applications for regional ARW cases, including the setup of different data types such as conventional, radiance, and GPSRO data and different analysis functions available in the GSI such as hybrid analysis.
\item[Chapter 6] illustrates the GSI applications for  global case and chemical cases.
\item[Appendix A] introduces the community tools available for GSI users.
\item[Appendix B] is content of the GSI namelist section OBS\_INPUT.
\item[Appendix C] contains a complete list of the GSI namelist with explanations and default values.
\end{description}

%\begin{description}
%\item[Chapter 1:] Overview
%\item[Chapter 2:] Software Installation
%\item[Chapter 3:] Running GSI
%\item[Chapter 4:] GSI Diagnostics and Tuning
%\item[Chapter 5:] GSI Applications for Regional 3Dvar and Hybrid
%\item[Appendix A:] GSI Community Tools
%\item[Appendix B:] Content of Namelist Section OBS\_INPUT
%\item[Appendix C:] GSI Namelist: Name, Default value, Explanation
%\end{description}

For the latest version of GSI User's Guide and released code, please visit the GSI User\textquotesingle s Website:
\begin{center}
\url{http://www.dtcenter.org/com-GSI/users/index.php}
\end{center}

Please send questions and comments to the GSI help desk:
\begin{center}
gsi-help@ucar.edu
\end{center}

This document and the annual GSI releases are made available through a community GSI effort jointly led by the Developmental Testbed Center (DTC) and the National Centers for Environmental Prediction (NCEP) Environmental Modeling Center (EMC), in collaboration with other GSI developers. To help sustain this effort, we recommend for those who use the community released GSI, the GSI helpdesk, the GSI User's Guide, and other DTC GSI services, please refer to this community GSI effort in their work and publications. 

For referencing this user's guide, please use:

\texttt{Hu, M., H. Shao, D. Stark, K. Newman, C. Zhou, and X. Zhang, 2016: Grid-point Statistical
Interpolation (GSI) User's Guide Version 3.5. Developmental Testbed Center. Available at
http://www.dtcenter.org/com-GSI/users/docs/index.php, 141 pp.}

For referencing the general aspect of the GSI community effort, please use:

\texttt{Shao, H., J. Derber, X.-Y. Huang, M. Hu, K. Newman, D. Stark, M. Lueken, C. Zhou, L. Nance, Y.-H. Kuo, B. Brown, 2016: Bridging Research to Operations Transitions: Status and Plans of Community GSI. Bulletin of the American Meteorological Society, doi:10.1175/BAMS-D-13-00245.1, in press}


\end{flushleft}
\end{titlepage}
\pagebreak{}




