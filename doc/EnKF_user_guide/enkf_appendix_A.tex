\chapter{Content of Namelist}\label{nameless}

The following are lists and explanations of the EnKF namelist variables. Users can also check file \textit{params.f90} for the details. \\

Section \textbf{nam\_enkf}
\begin{table}[htbp]
\centering
\begin{tabular}{p{3cm}p{7cm}p{2.5cm}p{1.5cm}}
\hline
Variable Name&Description&Data Type&Default\\
\hline
datein&Analysis date in YYYYMMDDHH&integer&0\\
datapath&path to data directory (include trailing slash)&Character (len=500) &""\\
iassim\_order&= 0 for the order they are read in,\newline
=1 for random order\newline
= 2 for order of predicted posterior variance reduction (based on prior) &integer&0\\
covinflatemax&maximum inflation&real(r\_single)&1.e30\\
covinflatemin&minimum inflation&real(r\_single)&1.0\\
deterministic&if true, use EnSRF w/o perturbed obs.\newline
                     if false, use perturbed obs EnKF. & logical & true\\
sortinc&if false, re-order obs to minimize regression 
                     errors as described in Anderson (2003). &logical &true\\
corrlengthnh&length for horizontal localization (in km) in north hemisphere&real(r\_single)&2800\\
corrlengthtr&length for horizontal localization (in km) in tropic&real(r\_single)&2800\\
corrlengthsh&length for horizontal localization (in km) in south hemisphere&real(r\_single)&2800\\
\hline
\end{tabular}
\end{table} 


\begin{table}[htbp]
\centering
\begin{tabular}{p{3cm}p{7cm}p{2.5cm}p{1.5cm}}
\hline
Variable Name&Description&Data Type&Default\\
\hline
varqc&Turn on varqc & logical & false \\
huber&use huber norm instead of "flat-tail" &logical & fales\\
nlons&number of lons &integer&0\\
nlats&Number of lats & integer & 0\\
 smoothparm&smoothing parameter for inflation (-1 for no smoothing) & real(r\_single) & -1\\
readin\_localization&If true, read in localization length scales from an external file & logical & false\\
zhuberleft&Parameter for "huber norm" QC  & real(r\_single) & 1.e30\\
  zhuberright&Parameter for "huber norm" QC & real(r\_single) &1.e30\\
   obtimelnh&observation time localization in hours over north hemisphere & real(r\_single) &25.925\\
obtimeltr&observation time localization in hours over tropic & real(r\_single) & 25.925\\
    obtimelsh& observation time localization in hours over south hemisphere & real(r\_single) &25.925\\
    
    reducedgrid & Do smooth in a reduced grid with a variable number 
on longitudes per latitude. The number of longitudes is chosen so that the zonal grid
spacing is approximately the same as at the equator & logical & false \\
lnsigcutoffnh & length for vertical localization in ln(p) over north hemisphere for conventional observation&real(r\_single)& 2.0\\
lnsigcutofftr &length for vertical localization in ln(p) over tropic conventional observation&real(r\_single)&2.0\\
lnsigcutoffsh &length for vertical localization in ln(p) over south hemisphere for conventional observation&real(r\_single) &2.0\\
lnsigcutoffsatnh &length for vertical localization in ln(p) over north hemisphere for satellite radiance observation&real(r\_single)& -999.0\\
 lnsigcutoffsattr &length for vertical localization in ln(p) over tropic satellite radiance observation&real(r\_single)& -999.0\\
 lnsigcutoffsatsh  &length for vertical localization in ln(p)over south hemisphere for satellite radiance observation&real(r\_single) &-999.0\\
 lnsigcutoffpsnh&length for vertical localization in ln(p) over north hemisphere for surface pressure observation&real(r\_single) &-999.0\\
  lnsigcutoffpstr &length for vertical localization in ln(p) over tropic surface pressure observation&real(r\_single)& -999.0\\
  lnsigcutoffpssh &length for vertical localization in ln(p) over south hemisphere for surface pressure observation&real(r\_single) &-999.0\\
\hline
\end{tabular}
\end{table} 



 
         
 
\begin{table}[htbp]
\centering
\begin{tabular}{p{3cm}p{7cm}p{2.5cm}p{1.5cm}}
\hline
Variable Name&Description&Data Type&Default\\
\hline
  analpertwtnh&adaptive posterior inflation parameter over north hemisphere:\newline
               1 means inflate all the way back to prior spread&real(r\_single)& 0.0\\    
 analpertwtsh & adaptive posterior inflation parameter over tropic:\newline
1 means inflate all the way back to prior spread&real(r\_single) &0.0\\
  analpertwttr&adaptive posterior inflation parameter over south hemisphere:\newline
1 means inflate all the way back to prior spread&real(r\_single) &0.0\\
sprd\_tol&tolerance for background check:
observations are not used if they are more than sqrt(S+R) from mean, 
  where S is ensemble variance and R is observation error variance. &real(r\_single) &9.9e31\\
nlevs &total number of levels&integer &0\\
  nanals&number of ensemble members&integer & 0\\
   nvars&number of 3d variables to update. For hydrostatic models, typically 5 (u,v,T,q,ozone).&integer &5\\
saterrfact &factor to multiply sat radiance errors&real(r\_single) &1.0\\
   univaroz &If true, ozone observations only affect ozone &logical &true\\
   regional &If true, analysis is for regional&logical &false\\
   use\_gfs\_nemsio&If true, GFS background is in NEMS format&logical& false\\
   paoverpb\_thresh &if observation space posterior variance divided by prior variance less than this value, 
observation is skipped during serial processing. \newline
1.0 = don't skip any obs  &(r\_single) &1.0\\
latbound &definition of tropics and mid-latitudes (for inflation). &real(r\_single) &25.0\\
delat &width of transition zone &real(r\_single)& 10.0\\
pseudo\_rh &use 'pseudo-rh' analysis variable, as in GSI &logical & false\\
numiter&number of times to iterate state/bias correction update. (only relevant when satellite radiances assimilated, i.e. nobs\_sat>0)&integer &1.0\\
\hline
\end{tabular}
\end{table} 



\begin{table}[htbp]
\centering
\begin{tabular}{p{3cm}p{7cm}p{2.5cm}p{1.5cm}}
\hline
Variable Name&Description&Data Type&Default\\
\hline
biasvar&background error variance for rad bias coeffs (used in radbias.f90). Default is (old) GSI value.\newline
if negative, bias coeff error variace is set to - biasvar/N, where N is number of obs per instrument/channel.\newline
if newpc4pred is .true., biasvar is not used - the estimated analysis error variance from the previous cycle is 
used instead (same as in the GSI). &real(r\_single) &0.1\\
lupd\_satbiasc&if performing satellite bias correction update&logical &true\\
 cliptracers&if true, tracers are clipped to zero when read in, and just before they are written out.&logical& true\\
simple\_partition &partition obs for enkf using Graham's rule&logical &true\\
 adp\_anglebc&turn off or on the variational radiance angle bias correction&logical& false\\
angord &order of polynomial for angle bias correction&Integer& 0\\
 newpc4pred&controls preconditioning due to sat-bias correction term&logical&\\
nmmb&If true, ensemble forecast is NMMB&logical &false\\
iau&&logical &false\\
nhr\_anal&background forecast time for analysis&integer &6\\
  letkf\_flag&If true, do LETKF&logical& false\\
   boxsize &Observation box size for LETKF (deg)&real(r\_single) &90.0\\
   massbal\_adjust&mass balance adjustment for GFS&logical &false\\
    use\_edges&logical to use data on scan edges (.true.=to use)&logical &true\\
     emiss\_bc&If true, turn on emissivity bias correction&logical& false\\
\hline
\end{tabular}
\end{table} 

\pagebreak[0]

\begin{table}[htbp]
\centering
\begin{tabular}{p{3cm}p{7cm}p{2.5cm}p{1.5cm}}
&Section \textbf{nam\_wrf} &&\\
\hline
Variable Name&Description&Data Type&Default\\
\hline
arw &regional dynamical core ARW&logical &false\\ 
nmm&regional dynamical core NMM&logical& true\\
doubly\_periodic&&logical&true\\
\hline
\end{tabular}
\end{table} 

\begin{table}[htbp]
\centering
\begin{tabular}{p{3cm}p{6cm}p{3cm}p{1.5cm}}
& Section \textbf{satobs\_enkf}&&\\
\hline
Variable Name&Description&Data Type&Default\\
\hline
sattypes\_rad&strings describing the satellite data type (which form part of the diag* filename).& 
character(len=20) array (nsatmax\_rad) & '"" \\
dsis&strings corresponding to sattypes\_rad which correspond to the names in the NCEP global\_satinfo file.&
character(len=20) array (nsatmax\_rad)&""\\
\hline
\end{tabular}
\end{table} 

\begin{table}[htbp]
\centering
\begin{tabular}{p{3cm}p{6cm}p{3cm}p{1.5cm}}
&Section \textbf{ozobs\_enkf}&&\\
\hline
Variable Name&Description&Data Type&Default\\
\hline
sattypes\_oz&strings describing the ozone satellite data type (which form part of the diag* filename)&
character(len=20) array (nsatmax\_oz)&""\\
\hline
\end{tabular}
\end{table} 
