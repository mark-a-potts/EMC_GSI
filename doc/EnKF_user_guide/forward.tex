\begin{titlepage}
%\BgThispage
%\newgeometry{left=1cm,right=4cm}
\vspace*{0.5cm}
\noindent

\begin{flushleft}
\textcolor{darkgray}{\LARGE Forward}
\vspace*{1cm}\par

This User\textquotesingle s Guide for the community ensemble Kalman filter (EnKF) data analysis system is particularly geared for beginners. 
It describes the fundamentals of using EnKF, including basic skills of installing, running, diagnosing, and tuning EnKF. EnKF version (v) 1.1 was released in July 2016. This version of code is compatible with the Gridpoint Statistical Interpolation (GSI) analysis system community release v3.5.

This User\textquotesingle s Guide includes six chapters and one appendix:

\begin{description}
\item[Chapter 1] provides a background introduction of the EnKF operational and community system, EnKF review committee, and data types that can be used in this version.
\item[Chapter 2] contains basic information about how to get started with EnKF, including system requirements; required software (and how to obtain it); how to download EnKF; and information about compilers, libraries, and how to build the code.
\item[Chapter 3] focuses on the input files needed to run EnKF and how to configure and run GSI observer and EnKF through a sample run script. This chapter also provides an example of a successful EnKF run.
\item[Chapter 4] includes information about diagnostics and tuning of the EnKF system through EnKF standard output and namelist variables.
\item[Chapter 5] illustrates how to setup and run the GSI observer and EnKF for a regional configuration and a global configuration, as well as how to diagnose the results.
\item[Chapter 6] introduces EnKF theory and the main structure of the code. 
\item[Appendix A] describes the contents of the EnKF namelist.
\end{description}

This document is updated annually. For the latest version of this document and annual released code, please visit the EnKF User\textquotesingle s Website:
\begin{center}
 \url{http://www.dtcenter.org/EnKF/users/index.php}
 \end{center}
 Please send questions and comments to the EnKF help desk:
\begin{center}
enkf-help@ucar.edu
\end{center}
This document and the annual EnKF releases are made available through a community EnKF effort led by the Developmental Testbed Center (DTC), in collaboration with EnKF developers. To help sustain this effort, we encourage for those who use the community released EnKF, the EnKF helpdesk, the EnKF User's Guide, and the other DTC EnKF services, please refer to this user's guide in their work and publications.

%need update page number

\textbf{Citation:}\\
\texttt{Liu, H., M. Hu, D. Stark, H. Shao, K. Newman, and J. Whitaker, 2016: Ensemble Kalman Filter (EnKF) User\textquotesingle s Guide Version 1.1. Developmental Testbed Center. Available at \url{http://www.dtcenter.org/EnKF/users/docs/index.php}, 80 pp.}

\end{flushleft}
\end{titlepage}
\pagebreak{}




